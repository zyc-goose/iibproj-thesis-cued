%!TEX root = ../thesis.tex

%%%%% Chapter: Conclusion %%%%%
\chapter{Conclusion}
\label{chap:conclusion}

\ifpdf
    \graphicspath{{Chapter10/Figs/Raster/}{Chapter10/Figs/PDF/}{Chapter10/Figs/}}
\else
    \graphicspath{{Chapter10/Figs/Vector/}{Chapter10/Figs/}}
\fi



\section{Summary of the Tesseract OCR engine}

The Tesseract OCR engine is a straightforward tool to convert handout images to chunks of extracted text. It contains a built-in Page Layout Analysis which identifies distinct chunks of text information. The Tesseract PLA provides 5 different scales: word, line, paragraph, block and page. The alignment accuracy is roughly proportional to the size of the scale.

Recognising handwritten text with OCR engine is generally hard, and fine-tuning the original Tesseract model with external handwritten data also didn't work quite well, according to the evaluation results in \Cref{chap:eval-results}. 

The Tesseract OCR engine has done a very decent job in general, with the only obvious weak point that it cannot recognise handwritten data very well. Recognising handwritten text with decent accuracy would be very helpful to achieve even better alignment results, which is definitely worthwhile to be investigated in the future.

\section{Summary of the \texttt{diff}-based Alignment Algorithm}

The design of alignment algorithm of the system is based on the idea of the Unix command-line tool \texttt{diff}. The core part of the alignment algorithm in the system solves the optimal alignment between the sequence words in the handout and the sequence of words in the lecture audio file. The alignment algorithm could be parameterised by using different joint weight functions (JWF).

In the final alignment algorithm, two more constraints have been added on top of the baseline \texttt{diff} algorithm, which are the time constraint and the common-word penalty. These constraints have improved the final alignment accuracy but only by a little amount.


\section{Feasibility of the General Idea}

The project aims to test the feasibility of the idea of making a software system to provide direct links from chunks of the handouts to the cooresponding segments in the lecture audio files. A simple software system based on the Tesseract OCR engine, the Google Cloud speech recogniser and the \texttt{diff}-based alignment algorithm has been developed to investigate this idea. An evaluation framework has been designed and implemented to quantify the performance of the system. A GUI system has been developed to visualise the core elements in the system, including a visualisation of the bounding-boxes in the input handout and a playback system which takes the user from a specific chunk of handout to the matched parts in the audio file.

As a conclusion, this idea should be regarded as feasible based on the results from the evaluation framework. When the OCR scale is set to `paragraph', the best alignment accuracy that can be achieved is 46.67\%. Although this number does not seem to be high, the actual experience of using the alignment functionality of the GUI is actually quite satisfactory. For most of the time, the alignment can find the positions of the audio segments at roughly the right place and these segments usually have half of their total length intersected with the ground-truth segments (this agrees with the alignment accuracy 46.67\%). In practice we just need to move around the audio cursor by a little bit if the alignment could link us to the roughly correct position.

Furthermore, considering that we have used an extremely simple system architecture which roughly solves the task, it would be completely possible for us to achieve even better results using more sophisticated architectures. 